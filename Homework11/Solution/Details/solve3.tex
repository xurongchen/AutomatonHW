\begin{proof}
    % 基于K,S,DN,MP公理系统进行证明:
    根据希尔伯特公理系统进行证明:
    \begin{align*}
        \text{K公理:}\; &A \rightarrow (B \rightarrow A)\\
        \text{S公理:}\; &(A \rightarrow (B \rightarrow C))\rightarrow ((A\rightarrow B)\rightarrow(A\rightarrow C))\\
        % \text{DN:}\;&\neg \neg A \rightarrow A\\
        \text{MP规则:}\; &\{A\rightarrow B,A\} \models B.
    \end{align*}
    定理$\models A\rightarrow A$的证明:
    \begin{align*}
        \models&\; (A\rightarrow ((D\rightarrow A)\rightarrow A))\rightarrow \\
            &((A\rightarrow (D\rightarrow A))\rightarrow (A\rightarrow A))
            &\text{S}(B\Leftarrow D\rightarrow A,C\Leftarrow A)\; \text{1}\\
        \models&\; A\rightarrow ((D\rightarrow A)\rightarrow A)&\text{K}(B\Leftarrow D\rightarrow A)\; \text{2}\\
        \models&\; (A\rightarrow (D\rightarrow A))\rightarrow (A\rightarrow A) &\text{MP,(1),(2)}\; \text{3}\\
        \models&\; A\rightarrow (D\rightarrow A) &\text{K}\;\text{4}\\
        \models&\; A\rightarrow A&\text{MP,(3),(4)}\; \text{5}\\
    \end{align*}
    $\models (p\rightarrow q) \Rightarrow p\models q$的证明:
    \begin{align*}
        \models&\; p\rightarrow q  &\text{[前提]\;1}\\
        % \models&\; \neg \neg p\rightarrow p  &\text{[DD引入]\;2}\\
        % \models&\; p\rightarrow p  &\text{[定理$A\rightarrow A$]\;2}\\
        % p\models&\; p &\text{[MP\{2, $p$\}]\;3}\\
        p\models&\; q &\text{[MP\{1, $p$\}] 2}\\
    \end{align*}

    $p\models q \Rightarrow\; \models (p\rightarrow q) $的证明:

    假设$A\models B$基于演绎证明系统的一个证明序列为$B_1,B_2,\ldots,B_n$($B_n=B$)。
    
    下面采用归纳法证明$A\models B \Rightarrow \; \models (A\rightarrow B)$:

    \textbf{初始:} $n=1$。此时$B_1$只能有两个来源:

    \textit{初始情形1:}$B_1$是公理(包括公理的某种代入)。此时显然$\models B_1$,又根据K公理可知,$\models B_1\rightarrow (A \rightarrow B_1)$,
    应用MP规则,$\models (A \rightarrow B_1)$;

    \textit{初始情形2:}$B_1$即$A$。此时由于定理$\models A\rightarrow A$,可知$\models (A \rightarrow B_1)$;
    
    于是初始$\models (A \rightarrow B_1)$。

    \textbf{归纳:} $n=i$。此时$B_i$有两个来源:

    \textit{归纳情形1:}$B_i$是公理(包括公理的某种代入)或者$A$,于是同初始情形,可证$\models (A \rightarrow B_i)$;

    \textit{归纳情形2:}$B_i$是$B_j,B_m$应用MP规则的归结($j<i,m<i$),假设MP$\frac{B_j;B_m}{B_i}$,另外我们已知
    $\models (A \rightarrow B_j)$,以及$\models (A \rightarrow B_m)$。又根据MP规则,
    不是一般性地,我们可以假定$B_m$是$B_j\rightarrow B_i$的形式,于是代入公式可得
    $\models (A \rightarrow (B_j\rightarrow B_i))$。注意到,S公理的一个有效的代入是
    $\models (A \rightarrow (B_j \rightarrow B_i))\rightarrow ((A\rightarrow B_j)\rightarrow(A\rightarrow B_i))$,
    于是应用MP规则,得到$\models ((A\rightarrow B_j)\rightarrow(A\rightarrow B_i))$,继续利用$\models (A \rightarrow B_j)$
    应用MP规则,得到$\models (A \rightarrow B_i)$。

    于是归纳$\models (A \rightarrow B_i)$。

    综上所述,根据初始条件和归纳条件,对于任意长度的证明序列,$A\models B \Rightarrow\; \models (A \rightarrow B)$。

    双向得证。
\end{proof}