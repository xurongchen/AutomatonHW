\begin{proof}~\\
    $\phi_1,\phi_2,\ldots,\phi_n \vdash \psi \Rightarrow\; \models \phi_1\rightarrow\phi_2\rightarrow\ldots\rightarrow\phi_n\rightarrow\psi$:

    采用反证法。对于蕴涵$x\rightarrow y$非重言式,则必然存在一组赋值使得$x\rightarrow y$为\textbf{F},$x$为\textbf{T},$y$为\textbf{F}。
    继而对于$\phi_1\rightarrow\phi_2\rightarrow\ldots\rightarrow\phi_n\rightarrow\psi$为非重言式,易归纳出
    $\phi_1,\phi_2,\ldots,\phi_n$为\textbf{T},而$\psi$为\textbf{F}。显然,这与
    $\phi_1,\phi_2,\ldots,\phi_n \models \psi$矛盾,进而根据完备性与题设的$\phi_1,\phi_2,\ldots,\phi_n \vdash \psi$矛盾。

    $\models\phi_1\rightarrow\phi_2\rightarrow\ldots\rightarrow\phi_n\rightarrow\psi \Rightarrow\phi_1,\phi_2,\ldots,\phi_n\vdash\psi$:

    将$\phi_1,\phi_2,\ldots,\phi_n$作为前提。第一步,利用\textit{Modus ponens}规则推导可以得到
    $\phi_1\vdash \phi_2,\phi_3,\ldots,\phi_n \rightarrow \psi$;第$k$步,继续利用\textit{Modus ponens}规则得到
    $\phi_1,\phi_2,\ldots\phi_k\vdash \phi_{k+1},\phi_{k+2},\ldots,\phi_n \rightarrow \psi$;如此反复,在第$n$步时
    得证。
\end{proof}