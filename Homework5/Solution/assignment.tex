% Homework template for Automaton and Formal Logic
% UPDATE: September 20, 2019 by Xu Rongchen
\documentclass[a4paper]{article}
\usepackage{ctex}
\ctexset{
proofname = \heiti{证明} %% set proof name
}
\usepackage{amsmath, amssymb, amsthm}
% amsmath: equation*, amssymb: mathbb, amsthm: proof
\usepackage{moreenum}
\usepackage{mathtools}
\usepackage{url}
\usepackage{bm}
\usepackage{enumitem}
\usepackage{graphicx}
\usepackage{subcaption}
\usepackage{booktabs} % toprule
\usepackage[mathcal]{eucal}
\usepackage{longtable}
\usepackage[thehwcnt = 5]{iidef} % set homework count

\usepackage{tikz}
\usetikzlibrary{automata,trees}

\thecoursename{自动机与形式逻辑}
\theterm{2019年秋季学期}
% \hwname{作业}
\slname{\heiti{解}}
\begin{document}
\courseheader
\theusername{徐荣琛}
\thestuno{2019214518}
\theinstitute{软件学院}

\info

\begin{enumerate}
  \setlength{\itemsep}{3\parskip}
  %% Homework Start here:
  %% \item to enumerate the problem ID: Format as 'HomeworkID.ProblemID'
  %% \begin{solution} XXXX \end{solution} is to make a solution
  %% \begin{proof} XXXX \end{proof} is to make a proof
  %% Suggest to use \input{path} command
  \item \textbf{[Exercise 7.3.1]} Show that the CFL's are closed under the following 
operations:

b) The operation $L/a$, defined in Exercise 4.2.2. \textit{Hint:}
Again, start with a CNF grammar for L.
  \begin{solution}
    $abaabaaabaa, aaaabaaaa, baaaaabaa \in L^*$
\end{solution}
  \item \textbf{[Exercise 2.3.2]} Convert to a DFA the following NFA:
\begin{center}
    \begin{tabular}{r||l|l}
        & $0$           & $1$\\ \hline \hline
        $\rightarrow p$ & $\{q,s\}$     & $\{q\}$\\
        $*q$            & $\{r\}$       & $\{q,r\}$\\
        $r$             & $\{s\}$       & $\{p\}$\\
        $*s$            & $\emptyset$   & $\{p\}$
    \end{tabular}
\end{center}
  \begin{solution}
    e)对于任意的$m$,选取$w=0^{m-1}1$,则$ww=0^{m-1}10^{m-1}1$,显然$|ww|>m$。若将
    $ww$写作$xyz$,其中$|xy| \le m$且$|y| > 0$,则$y=0^{t}1$,其中$0 \le t < m$。
    则$xz = 0^{2m-2-t}1$,由于1出现的次数为奇数,必然不能被$ww$接收,故违背泵
    引理,原语言非正则语言。\\
    f) 对于任意的$m$,选取$w=0^{m-1}1$,则$ww^R=0^{m-1}1^20^{m-1}$,显然$|ww^R|>m$。若将
    $ww^R$写作$xyz$,其中$|xy| \le m$且$|y| > 0$,则$y=0^{t}1$,其中$0 \le t < m$。
    则$xz = 0^{m-1-t}10^{m-1}$,由于1出现的次数为奇数,必然不能被$ww^R$接收,故违背泵
    引理,原语言非正则语言。\\
\end{solution}
  \item \textbf{[Exercise 3.2.3]} Convert the following DFA to a regular expression, using the 
state-elimination technique of Section 3.2.2.
\begin{center}
    \begin{tabular}{r||c|c}
                            & $0$          & $1$   \\ \hline \hline
        $\rightarrow *p$    & $s$          & $p$   \\
        $q$                 & $p$          & $s$   \\
        $r$                 & $r$          & $q$   \\
        $s$                 & $q$          & $r$ 
        
    \end{tabular}
\end{center}
  \begin{solution}
c)构建满足条件的空栈接收PDA:$E=\{Q,\Sigma,\Gamma,\delta,q_0,Z_0\}$。

其中:$Q=\{q\}$,$\Sigma=\{0,1\}$,$\Gamma=\{Z,X,Y\}$,$q_0 = q$,$Z_0=Z$;\\
$\delta$的非空转移包括:\\
\begin{tabular}{ll}
    $\delta(q,\epsilon,Z) = \{(q,\epsilon)\}$
    &$\delta(q,\epsilon,Y) = \{(q,\epsilon)\}$\\
    $\delta(q,0,Z) = \{(q,XZ)\}$
    &$\delta(q,0,X) = \{(q,XX)\}$\\
    $\delta(q,1,X) = \{(q,Y)\}$
    &$\delta(q,1,Y) = \{(q,\epsilon)\}$
\end{tabular}
\end{solution}
  \item \textbf{[Exercise 2.2.5]} Give DFA's accepting the following languages 
over the alphabet $\{0,1\}$:\\
c) The set of strings that either begin or end (or both) with $01$.\\
d) The set of strings such that the number of $0$'s is divisible by five, 
and the number of $1$'s is divisible by 3.

  \begin{solution}b) $q_0011\vdash 1q_011\vdash 10q_11\vdash 101q_1B\vdash 101Bq_2B$
\end{solution}
  \item \textbf{思考题[Exercise 2.2.5]} Give DFA's accepting the following languages 
over the alphabet $\{0,1\}$:\\
a) The set of all strings such that each block of five consecutive 
symbols contains at least two $0$'s.\\
b) The set of strings whose tenth symbol from the right end is a $1$.

  \begin{solution}
a) 若原有文法为$G$,则新文法的语言$L(G')=L(G)-\{\epsilon\}$的产生式描述:
\begin{align*}
    S &\rightarrow 0A0 \mid 1B1 \mid BB \mid 00 \mid 11 \mid B\\
    A &\rightarrow C\\
    B &\rightarrow S \mid A\\
    C &\rightarrow S
\end{align*}
\end{solution}
b)
\begin{align*}
    S &\rightarrow 0A0 \mid 1B1 \mid BB \mid 00 \mid 11 \mid B\\
    A &\rightarrow 0A0 \mid 1B1 \mid BB \mid 00 \mid 11 \mid B\\
    B &\rightarrow 0A0 \mid 1B1 \mid BB \mid 00 \mid 11 \mid B\\
    C &\rightarrow 0A0 \mid 1B1 \mid BB \mid 00 \mid 11 \mid B\\
\end{align*}
c)
\begin{align*}
    S &\rightarrow 0A0 \mid 1B1 \mid BB \mid 00 \mid 11 \mid B\\
    A &\rightarrow 0A0 \mid 1B1 \mid BB \mid 00 \mid 11 \mid B\\
    B &\rightarrow 0A0 \mid 1B1 \mid BB \mid 00 \mid 11 \mid B\\
\end{align*}

CHECK ME!!!
  % \item \textbf{[Exercise 7.1.9]} Provide the inductive proofs needed to complete
the following theorems:

b) Both directions of Theorems 7.6, where we show the correctness
of the algorithm in Section 7.1.2 for detecting the reachable symbols.
  % \begin{solution}
b)算法只给出可达符号:\\
按照算法进行对顺序进行归纳:\\
(1)初始$S$是可达符号;\\
(2)对于可达符号集合中对任意符号$X$,显然对于其任意的推导式中的符号$Y$($X \Rightarrow \alpha Y \beta$),$Y$一定也是可达符号。\\
由(1)(2)可归纳出算法只给出可达符号。

算法可以给出所有可达符号:\\
对任意的可达符号$X$,不妨设$S \xRightarrow[G]{*} \alpha X\beta$。下面对
推导式长度进行归纳:\\
(1)推导式长度为0,此时$X=S$,显然$X$可以被给出;\\
(2)推导式长度为$n>0$,则存在$S \xRightarrow[G]{*} \alpha_0 Y\beta_0 \Rightarrow \alpha_0\alpha_1 X\beta_1\beta_0$,其中产生式
$Y\Rightarrow \alpha_1 X\beta_1$,显然若$Y$可以被给出(关于$S$的推导式长度为$n-1$),则$X$可以被给出。\\
由(1)(2)可归纳出算法可以给出所有可达符号。
\end{solution}
\end{enumerate}

\end{document}


%%% Local Variables:
%%% mode: late\rvx
%%% TeX-master: t
%%% End:
