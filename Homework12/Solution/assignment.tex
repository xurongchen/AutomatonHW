% Homework template for Automaton and Formal Logic
% UPDATE: September 20, 2019 by Xu Rongchen
\documentclass[a4paper]{article}
\usepackage{ctex}
\ctexset{
proofname = \heiti{证明} %% set proof name
}
\usepackage{amsmath, amssymb, amsthm}
% amsmath: equation*, amssymb: mathbb, amsthm: proof
\usepackage{moreenum}
\usepackage{mathtools}
\usepackage{url}
\usepackage{bm}
\usepackage{enumitem}
\usepackage{graphicx}
\usepackage{subcaption}
\usepackage{booktabs} % toprule
\usepackage[mathcal]{eucal}
\usepackage{longtable}
\usepackage[thehwcnt = 12]{iidef} % set homework count
\usepackage[figuresright]{rotating}

\usepackage{tikz}
\usetikzlibrary{automata,trees}

\thecoursename{自动机与形式逻辑}
\theterm{2019年秋季学期}
% \hwname{作业}
\slname{\heiti{解}}
\begin{document}
\courseheader
\theusername{徐荣琛}
\thestuno{2019214518}
\theinstitute{软件学院}

\info

\begin{enumerate}
  \setlength{\itemsep}{3\parskip}
  %% Homework Start here:
  %% \item to enumerate the problem ID: Format as 'HomeworkID.ProblemID'
  %% \begin{solution} XXXX \end{solution} is to make a solution
  %% \begin{proof} XXXX \end{proof} is to make a proof
  %% Suggest to use \input{path} command
  \item \textbf{[Exercise 5.1.1]} Design context-free grammar for the following languages:\\
b) The set $\{a^ib^jc^k \mid i \neq j or j \neq k\}$, that is, the set of strings
of $a$'s followed by $b$'s followed $c$'s, such that there are either a different
number of $a$'s and $b$'s or a different number of $b$'s and $c$'s, or both.
  \begin{solution}
    (1)约定谓词$P(x)$:$x$是乌鸦,$Q(x,y)$:$x$、$y$的黑度相同。
    $$\forall x,y (P(x)\wedge P(y)\rightarrow Q(x,y))$$
    (2)约定谓词$P(x)$:$x$是筵席,$Q(x)$:$x$能够结束。
    $$\forall x (P(x)\rightarrow Q(x))$$
    (3)约定谓词$P(x)$:$x$是金子,$Q(x)$:$x$能够闪光。
    $$\exists x (\neg P(x) \wedge Q(x))$$
    (4)约定谓词$P(x)$:$x$是奇数,$Q(x)$:$x$是素数。
    $$\exists x (\neg P(x) \wedge Q(x))$$
    (5)约定谓词$P(x)$:$x$是偶数,$Q(x)$:$x$是素数,$R(x,y)$:$x$、$y$相等。
    $$\forall x,y (P(x)\wedge Q(x)\wedge P(y)\wedge Q(y)\rightarrow R(x,y))$$
    (6)约定谓词$P(x)$:$x$是猫,$Q(x)$:$x$是动物。
    $$(\forall x (P(x)\rightarrow Q(x)))\wedge \neg(\forall x (Q(x)\rightarrow P(x)))$$
    (7)约定谓词$P(x)$:$x$ is mushroom,$Q(x)$:$x$ is purple,$R(x)$:$x$ is poisonous。
    $$\neg \exists (x P(x)\wedge Q(x)\wedge R(x))$$
    (8)约定谓词$P(x)$:$x$ is mushroom,$Q(x)$:$x$ is purple,$R(x,y)$:$x$, $y$ is same。
    \begin{align*}
        &\exists x,y(P(x)\wedge Q(x)\wedge P(y)\wedge Q(y)\wedge \neg R(x,y) \wedge \\
        &\forall z(P(z)\wedge Q(z)\rightarrow (R(x,z)\vee R(y,z))))
    \end{align*}
\end{solution}
  \item 设字母表为$\{0,1\}$. 给出接受所有倒数第3个符号是1的语言的DFA.
  \begin{solution}DFA转移表:
\begin{center}
    \begin{tabular}{r||l|l||r||l|l}
                        & $0$           & $1$   & ~             & $0$           & $1$\\ \hline \hline
        $\rightarrow p$ & $qs$          & $q$   & $*qs$         & $r$           & $pqr$\\
        $*q$            & $r$           & $qr$  & $*rs$         & $s$           & $p$\\
        $r$             & $s$           & $p$   & $*pqr$        & $qrs$         & $pqr$\\
        $*s$            & $\emptyset$   & $p$   & $*qrs$        & $rs$          & $pqr$\\
        $*qr$           & $rs$          & $pqr$ & $\emptyset$   & $\emptyset$   & $\emptyset$\\
        
    \end{tabular}
\end{center}
    % 如图所示:\\
    % \begin{tikzpicture}
        
    %     % \draw[help lines] (0,0) grid (10,-5);
    %     \node[state,initial,initial text={}] at (1,0) (p){$\{p\}$};
    %     \node[state,accepting,minimum size=40] at (5,0) (q){$\{q\}$};
    %     \node[state] at (7,0) (r){$\{r\}$};
    %     \node[state,accepting] at (1,-2) (qs){$\{q,s\}$};
    %     \node[state,accepting] at (1,-4) (qr){$\{q,r\}$};
    %     \node[state,accepting] at (1,-6) (pqr){$\{p,q,r\}$};
    %     \node[state,accepting] at (3,-4) (rs){$\{r,s\}$};
    %     \node[state,accepting] at (5,-4) (s){$\{s\}$};
    %     \node[state] at (7,-4) (e){$\emptyset$};

    %     \node[state,accepting] at (3,-2) (qrs){$\{q,r,s\}$};
    %     \node[state,accepting] at (5,-2) (pq){$\{p,q\}$};


    %     \path[->]
    %     (p) edge node {0} (qs)
    %     (p) edge node {1} (q)

    %     (qs) edge node {0} (r)
    %     (qs) edge node {1} (pqr)

    %     (q) edge node {0} (r)
    %     (q) edge node {1} (qr)

    %     (r) edge node {0} (s)
    %     (r) edge node {1} (p)

    %     (pqr) edge node {0} (qrs)
    %     (pqr) edge node {1} (pqr)

    %     ;
    % \end{tikzpicture}
\end{solution}
\end{enumerate}

\end{document}


%%% Local Variables:
%%% mode: late\rvx
%%% TeX-master: t
%%% End:
