\begin{solution}
    (a) $req$成立的状态包括左上和右上两个状态。
    
    对于左上状态,有两个后继状态(右上和右下),两个状态均满足$busy$;
    同理,对于右上状态,两个后继状态也均满足$busy$;

    于是$\textrm{G }(req \rightarrow \textrm{ F } busy)$成立。
    
    (b)根据题意,$busy$和$ready$是一个变量的两个枚举值。于是$\neg busy = ready$。
    由于所有初始状态$ready$成立,即$\neg busy$成立,于是$req \textrm{ U } \neg busy$
    成立,$\neg(req \textrm{ U } \neg busy)$不成立。
\end{solution}