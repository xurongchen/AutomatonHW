\begin{solution}
    a)可能存在歧义,理解题目语义为:字符串存在至少5个字母,且连续5个字母至少包含两个0。\\
    如图所示:\\\bigskip
    \begin{tikzpicture}
        \node[state] at (0,-1) (x){x};
        \node[state,accepting] at (2,-1) (3){3};
        \node[state,accepting] at (4,-1) (2){2};
        \node[state,accepting] at (6,-1) (1){1};
        \node[state,accepting] at (8,-1) (0){0};

        \node[state] at (2,1.5) (4-3){4-3};
        \node[state] at (4,1.5) (4-2){4-2};
        \node[state] at (6,1.5) (4-1){4-1};
        \node[state] at (8,1.5) (4-0){4-0};

        \node[state] at (0,4) (3-3){3-3};
        \node[state] at (2,4) (3-2){3-2};
        \node[state] at (4,4) (3-1){3-1};
        \node[state] at (6,4) (3-0){3-0};

        \node[state] at (0,6) (2-2){2-2};
        \node[state] at (2,6) (2-1){2-1};
        \node[state] at (4,6) (2-0){2-0};

        \node[state] at (0,8) (1-1){1-1};
        \node[state] at (2,8) (1-0){1-0};

        \node[state,initial,initial text={}] at (0,10) (0-0){0-0};
 

        \path[->]
        (0-0) edge node [above right] {0} (1-0)
        (0-0) edge node [left] {1} (1-1)

        (1-0) edge node [above right] {0} (2-0)
        (1-0) edge node [right,pos=0.2] {1} (2-1)

        (2-0) edge node [above right] {0} (3-0)
        (2-0) edge node [right,pos=0.2] {1} (3-1)

        (3-0) edge node [above right] {0} (4-0)
        (3-0) edge node [right,pos=0.2] {1} (4-1)

        (4-0) edge node [right] {0} (0)
        (4-0) edge node [below,pos=0.1] {1} (1)

        (1-1) edge node [below left,pos = 0.2] {0} (2-0)
        (1-1) edge node [left] {1} (2-2)

        (2-1) edge node [below,pos = 0.1] {0} (3-0)
        (2-1) edge node [left] {1} (3-2)

        (3-1) edge node [below,pos=0.1] {0} (4-0)
        (3-1) edge node [left,pos=0.7] {1} (4-2)    

        (4-1) edge node [below,pos=0.1] {0} (0)
        (4-1) edge node [below,pos=0.1] {1} (2)

        (2-2) edge node [below,pos=0.1] {0} (3-0)
        (2-2) edge node [left] {1} (3-3)      

        (3-2) edge node [below,pos=0.1] {0} (4-0)
        (3-2) edge node [left] {1} (4-3)

        (4-2) edge node [below,pos=0.1] {0} (0)
        (4-2) edge node [below right,pos=0.4] {1} (3)

        (3-3) edge node [below,pos=0.1] {0} (4-0)
        (3-3) edge node [left] {1} (x)

        (4-3) edge node [below,pos=0.1] {0} (0)
        (4-3) edge node [below,pos=0.4] {1} (x)

        (0) edge [loop right] node {0} (0)
        (0) edge node [above] {1} (1)

        (1) edge [bend right] node [below] {0} (0)
        (1) edge node [above] {1} (2)

        (2) edge [bend right=50] node [below] {0} (0)
        (2) edge node [above] {1} (3)

        (3) edge [bend right=60] node [below] {0} (0)
        (3) edge node [above] {1} (x)

        (x) edge [loop below] node {0,1} (x)
        ;

     \end{tikzpicture}\\
     \clearpage
    b)构造自动机$A=\{Q,\Sigma,\delta,q_{start},Q_{end}\}$:\\
    $Q=\{q_0,q_1,\cdots,q_{1023}\}$\\
    $\Sigma=\{0,1\}$\\
    
    $q_{start}=q_0$\\
    $Q_{end}=\{q_i \mid i ~ \& ~ 512 = 512 \}$

    $\delta(q_k,m)=q_t$ 当且仅当 $((k << 1)) ~ | ~ m) ~ \& ~ 1023 = t$\\
    其中:\\
    \qquad$\&$是二进制位运算与;\\
    \qquad$|$是二进制位运算或;\\
    \qquad$<<$是二进制左移位运算.
\end{solution}