\begin{solution}
    Tape-1最初为输入且以$T$结尾,Tape-2为$Z_0$表示栈底元素、Tape-3为$Y_0$表示输出的开始位置。最初状态为$q_0$。
    下图是初始状态的三个Tape相对位置关系的一个示意:
    \begin{center}
        \begin{tabular}{r||c|c|c|c|c|c|c|c|c|c|c}\hline
            Tape-1& $\cdots$&$B$&$B$ & $a$& $a$& $b$& $\uparrow$& $b$&$T$&$B$ & $\cdots$ \\\hline
            Tape-2& $\cdots$&$B$&$Z_0$ &$B$ &$B$ &$B$ &$B$ &$B$&$B$&$B$  &$\cdots$\\\hline
            Tape-3& $\cdots$&$B$&$Y_0$ &$B$ &$B$ &$B$ &$B$ &$B$&$B$&$B$  &$\cdots$\\\hline
        \end{tabular}
    \end{center}
    $q_0$状态从Tape-1中读一个字符是$T$,则右移,且字符不变,进入接收状态$q_r$;若读一个字符$x$是$a$或者$b$,则将字符$x$变为$B$,并右移加入状态$q_{1a}$或
    $q_{1b}$;若读一个字符是$\uparrow$,则将字符$\uparrow$变为$B$,左移,进入状态$q_5$。

    对于$q_{1a}$,Tape-2上如果字符为$B$,则左移,且字符、状态不变;
    如果字符非为$B$,则进入$q_{2a}$,字符不变,右移。
    (此时在栈顶位置)$q_{2a}$随后Tape-2将元素从$B$变为$a$,并进入$q_{3}$,右移。\\
    对于$q_{1b}$,Tape-2上如果字符为$B$,则左移,且字符、状态不变;
    如果字符非为$B$,则进入$q_{2b}$,字符不变,右移。
    (此时在栈顶位置)$q_{2b}$随后Tape-2将元素从$B$变为$b$,并进入$q_{3}$,右移。

    $q_{3}$状态,根据Tape-1的内容,对于字符为$B$,右移、保持字符、状态不变;
    对于其它字符,左移保持字符不变,进入$q_{4}$。\\
    $q_{4}$状态,对于任意字符,右移、保持字符不变进入$q_{0}$。

    
    对于$q_5$,Tape-2上如果字符为$B$,则左移,且字符、状态不变;
    如果字符为$a$,则进入$q_{6a}$,字符$a$变$B$,左移;如果字符为$b$,则进入$q_{6b}$,字符$b$变$B$,左移;如果字符为$Z_0$,则进入$q_e$,字符不变,左移;

    对于$q_{6a}$,Tape-2上如果字符不为$Z_0$,则左移,且字符、状态不变;如果字符为$Z_0$,则进入$q_{7a}$,字符不变,右移。\\
    对于$q_{7a}$,Tape-3上如果字符不为$B$,则右移,且字符、状态不变;如果字符为$B$,则字符变为$a$,右移,进入$q_{3}$状态。

    对于$q_{6b}$,Tape-2上如果字符不为$Z_0$,则左移,且字符、状态不变;如果字符为$Z_0$,则进入$q_{7b}$,字符不变,右移。\\
    对于$q_{7b}$,Tape-3上如果字符不为$B$,则右移,且字符、状态不变;如果字符为$B$,则字符变为$b$,右移,进入$q_{3}$状态。

    \textbf{总结:}$q_0$表示一个操作周期结束下一个周期未开始之前的状态;
    $q_1x$表示读到压栈$x$指令的状态,当前位置未确定指向栈顶元素前的状态;
    $q_2x$表示读到压栈$x$指令的状态,当前位置确定指向栈顶第一个空位置的状态;
    $q_3$表示当前位置未确定指向下一个操作符号前的状态;
    $q_4$表示当前位置确定指向下一个操作符号左侧位置的状态;
    $q_5$表示读到出栈指令,当前位置未指向栈顶元素前的状态;
    $q_{6x}$表示读到出栈指令,并已经弹出元素$x$,当前位置未到达栈底(输出开始位置)的状态;
    $q_{7x}$表示读到出栈指令,并已经弹出元素$x$,当前位置未到达下一个输出空位置的状态;
    $q_r$表示输入序列正常终止的结束状态;
    $q_e$表示对空栈进行弹出操作导致的错误状态;
\end{solution}