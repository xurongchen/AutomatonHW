\textbf{[Exercise 8.4.7]} In this exercises, we shall implement a stack using a special
3-tape TM.

1. The first tape will be used only to hold and read the input. The input
alphabet consists of the symbol $\uparrow$, which we shall interpret as "pop the
stack," and the symbols $a$ and $b$, which are interpreted as "push an $a$
(respectively $b$) onto the stack."

2. The second tape is used to store the stack.

3. The third tape is the output tape. Every time a symbol is popped from
the stack, it must be written on the output tape, following all previously
written symbols.

The Turing machine is required to start with an empty stack and implement the
sequence of push and pop operations, as specified on the input, reading from
left to right. If the input causes the TM to try to pop and empty stack, then it
must halt in a special error state $q_e$. If the entire input leaves the stack empty
at the end, then the input is accepted by going to the final state $q_r$. Describe
the transition function of the TM informally but clearly. Also, give a summary
of the purpose of each state you use.
