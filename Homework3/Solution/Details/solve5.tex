\begin{proof}
    b) \\
    (1) 证明$(RS+R)^*R \subseteq R(SR+R)^*$:\\
    设$T = (RS+R)^*R$。如果对$i \ge 0$,有$T_i = (RS+R)^iR$,那么$T$有如下表示:
    $$T = \bigcup_{i=0}^{\infty} T_i$$
    显然,$T_0 = R \subseteq R(SR+R)^*$;\\
    假设$T_i \subseteq R(SR+R)^*$,则:
    \begin{align*}
        T_{i+1} =& (RS+R)T_i \\
        \subseteq& (RS+R)R(SR+R)^* \\
        =& R(S+\epsilon)R(SR+R)^* \\
        =& R(SR+R)(SR+R)^* \\
        \subseteq& R(SR+R)^*
    \end{align*}
    所以,对任意$i \ge 0$,有$T_i \subseteq R(SR+R)^*$。\\
    于是得到$T \subseteq R(SR+R)^*$.\\
    \medskip
    (2) 证明$R(SR+R)^* \subseteq (RS+R)^*R$:\\
    设$U = R(SR+R)^*$。如果对$i \ge 0$,有$U_i = R(SR+R)^i$,那么$U$有如下表示:
    $$U = \bigcup_{i=0}^{\infty} U_i$$
    显然,$U_0 = R \subseteq (RS+R)^*R$;\\
    假设$U_i \subseteq (RS+R)^*R$,则:
    \begin{align*}
        U_{i+1} =& U_i(SR+R) \\
        \subseteq& (RS+R)^*R(SR+R) \\
        =& (RS+R)^*R(S+\epsilon)R \\
        =& (RS+R)^*(RS+R)R \\
        \subseteq& (RS+R)^*R
    \end{align*}
    所以,对任意$i \ge 0$,有$U_i \subseteq (RS+R)^*R$。\\
    于是得到$U \subseteq (RS+R)^*R$.\\
    \medskip
    (3) 综上,$(RS+R)^*R = R(SR+R)^*$\\
    \bigskip
    d) $(R+S)^*S \neq (R^*S)^*$\\
    反例:若$S$满足$\epsilon \notin S$,则$\epsilon \notin (R+S)^*S$且
    $\epsilon \in (R^*S)^*$,所以两者不等。
\end{proof}